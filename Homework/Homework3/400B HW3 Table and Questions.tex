
\documentclass{article}
\usepackage{graphicx} % Required for inserting images
\usepackage{longtable}


\begin{document}

\title{400B Mass Table and Questions}
\author{Ethan Figureido}
\date{February 2025}

\maketitle
   

\begin{longtable}{|p{1.8cm}|p{1.8cm}|p{1.8cm}|p{1.8cm}|p{1.8cm}|p{1.8cm}|p{1.8cm}|}
\hline
\multicolumn{7}{|c|}{Mass Break Down of the local group} \\
\hline
Galaxy Name & Halo Mass 10$^{12}$\(M_\odot\) & Disk Mass 10$^{12}$\(M_\odot\) & Bulge Mass 10$^{12}$\(M_\odot\) & Total Mass 10$^{12}$\(M_\odot\) & Stellar Mass Fraction & Local Group Total Mass 10$^{12}$\(M_\odot\)  \\
\hline
MW & 1.97 & 0.075 & 0.01 & 2.06 & 0.041 & -\\
\hline
M31 & 1.92 & 0.12 & 0.019 & 2.06 & 0.068 & - \\
\hline
M33  & 0.187 & 0.0093 & none & 0.196 & 0.047 & - \\
\hline
LG Total Mass 10$^{12}$\(M_\odot\) &-&-&-&-&-&4.315 \\
\hline

\end{longtable}


\section{Questions}


1. In this simulation, the total masses of the Milky Way (MW) and M31 are approximately the same, sitting at 2.06*10$^{12}$\(M_\odot\). The Halo is the dominant component in the total mass.
\\

2. One can see that M31 has a greater stellar mass than MW, in both the bulge and disk portions of the galaxy. Thus M31 will have a greater luminosity, as it has more massive stars which dominate galaxy luminosity.
\\

3. Although it has a lower stellar mass, MW has the larger total dark matter mass. This is surprising, as I expected M31 to have a higher dark matter mass to pair with its higher stellar mass.
\\

4. The baryon fractions for MW, M31, and M33 are 4.1\%, 6.8\% and 4.7\%, respectively. They are approximately the same for each galaxy and much smaller than the baryon fraction of the universe of 16\%. These fractions may be different because the baryon fraction of the universe includes all of the mass in the intergalactic medium. That is, the baryon fraction of the universe includes all the gas sitting between galaxies.



\end{document}
